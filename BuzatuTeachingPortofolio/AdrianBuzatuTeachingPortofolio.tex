%%%%%%%%%%%%%%%%%%%%%%%%%%%%%%%%%%%%%%%%%
% Simple Sectioned Essay Template
% LaTeX Template
%
% This template has been downloaded from:
% http://www.latextemplates.com
%
% Note:
% The \lipsum[#] commands throughout this template generate dummy text
% to fill the template out. These commands should all be removed when 
% writing essay content.
%
%%%%%%%%%%%%%%%%%%%%%%%%%%%%%%%%%%%%%%%%%

%----------------------------------------------------------------------------------------
%	PACKAGES AND OTHER DOCUMENT CONFIGURATIONS
%----------------------------------------------------------------------------------------

\documentclass[12pt]{article} % Default font size is 12pt, it can be changed here

\usepackage{geometry} % Required to change the page size to A4
%\geometry{a4paper} % Set the page size to be A4 as opposed to the default US Letter
%\usepackage[margin=0.5in]{geometry}
%addtolength{\oddsidemargin}{-.875cm}
%\addtolength{\evensidemargin}{-.875cm}
%\addtolength{\textwidth}{1.75cm}
%\addtolength{\topmargin}{-.875cm}
%\addtolength{\topmargin}{-.875cm}
%\addtolength{\textheight}{1.75cm}


\usepackage{geometry}
 \geometry{
 a4paper,
 total={210mm,297mm},
 left=20mm,
 right=20mm,
 top=20mm,
 bottom=20mm,
 }




\usepackage{graphicx} % Required for including pictures

\usepackage{float} % Allows putting an [H] in \begin{figure} to specify the exact location of the figure
\usepackage{wrapfig} % Allows in-line images such as the example fish picture

\usepackage{lipsum} % Used for inserting dummy 'Lorem ipsum' text into the template

\usepackage[title,titletoc,toc]{appendix} %appendix

\usepackage{url}

\linespread{1.2} % Line spacing

%\setlength\parindent{0pt} % Uncomment to remove all indentation from paragraphs

\graphicspath{{Pictures/}} % Specifies the directory where pictures are stored

\begin{document}

{
\center% Center everything on the page
\textsc{\LARGE Teaching Portofolio}\\[0.5cm] % Name of your university/college
\textsc{\LARGE Adrian Buzatu}\\[1.0cm] % Major heading such as course name
\textsc{\large McGill University, Canada - MSc \& PhD}\\[0.5cm] % Minor heading such as course title
\textsc{\large University of Glasgow, UK - postdoc}\\[1.0cm] % Minor heading such as course title
{\large \today}\\[0.5cm] % Date, change the \today to a set date if you want to be precise

}



%----------------------------------------------------------------------------------------
%	TABLE OF CONTENTS
%----------------------------------------------------------------------------------------

\tableofcontents % Include a table of contents

%\newpage % Begins the essay on a new page instead of on the same page as the table of contents 

%----------------------------------------------------------------------------------------
%	INTRODUCTION
%----------------------------------------------------------------------------------------

\section{Teaching approach} % Major section
%\subsection{Intentionality of teaching and supervision}


\subsection{Conceptions that drive your approach to teaching}

Traditional teaching is teacher-centred and not student-centred. Teachers present information, perhaps add an example, and move on to new information, in a continuous loop. The student absorbtion of information is not checked in real time, but delayed until the final exam, when it is too late. During my MSc and PhD at McGill University in Canada, I was taught about a student-centered interacting teaching process~\cite{T-PULSE}, which I have integrated in my teaching and supervision practice ever since. Students are engaged and take an interactive role in the learning process. Information is constructed, not received. As information does not arrive in a vacuum, but adds to the already existing sets of world beliefs, its important for a teacher to understand the background beliefs and challenge them. 

\subsection{Learning goals you have for students}

When teaching or supervising research, my goals are to help students become excellent scientists, able to work autonomously or in groups. This is achieved by understanding the scientific method, having the mathematical, physics and computing, and soft skills necessary to answer scientific questions, to interpret and present results efficiently. It is essential they feel confident to be able to tackle complex problems, as they can be deconstructed from simple problems already familiar from our teaching or supervision. I also emphasize the importance of history of our field and context of the research topic. Having the background and the skills, I further train the students to know where and how to search for further information. 

\subsection{Why certain teaching methods are used}

Memories form and are strengthened in at least two ways: by repetition and by verbalising. It is well known that one remembers about 50\% of what one sees or hears, 70\% of what one rephrases in one's own words, or in another language, and 90\% of what one teaches somebody else, since while teaching one both verbalizes and repeats. It is for this reason that teaching is very helpful for professors to learn. What is very important is that students learn too. Students need to be encouraged to teach as well, albeit in a subtle indirect way. 

\ \\This is achieved by asking a questionnaire before the teaching about the background of each student and then arrange for the students to work in groups as heterogenous as possible. On an academic side, the most capable students explain to the others, bringing them up to speed. This way, the smart students are not disadvantaged, as believed, but rather they strengthen even more their pedagogical skills and they understand even more the material they have explained. On a human side, one should aim to place at least two females or two foreign students together in the same group, so they do not feel isolated and feel free to contribute to the group work. Shy students also benefit from the group, as they may feel more comfortable to talk to a small group of peers while being afraid to talk in front of the whole class and the teacher. If a large material is divided between groups and then a representative of each group summarizes it to the entire class, effectively the classes covers more material in the same amount of time and the person explaining understands the topic even more. 

\ \\Acknowledging that the attention span is of about 10 minutes, one needs to change the rythm after that, by using a one minute question, or game, or exercise, or joke, or even a quick dance, and then start anew. 

\ \\During teaching I required the students to use the same advanced tools they would use later on in their career. For obtaining results, in programming we use Unix or MacOS, Python and C++. In order to communicate efficiently scientific results, the structure of presentations and posters is constructed using using Jean-luc Dumont's advice~\cite{JeanLucDumontVideo}~\cite{JeanLucDumontBook}, while for the technical side for reports we use LaTex instead of Word, while for presentations and posters we allow for either Beamer (LaTex) or Powerpoint. 

\ \\For research supervision, I make sure the students are able to solve the basic problems, in either physics or programming, so that they can deconstruct later complex problems into these basics steps they already know. As after they give it a first try, I offer my step-by-step solutions or programming code, so they can learn by a clear example. 

%\subsection{Why you evaluate learning as you do}

\subsection{Directions and plans for teaching}

I would love to continue to teach physics skills, science communication skills, as well as programming for science and particle physics at either BSc, MSc or PhD level, and do so ever more efficiently. I would love to teach an experimental particle physics class, and I am open to teaching other physics classes as well. 

\section{Teaching responsibilities} % Major section

\subsection{Teaching physics skills}

In the summer of 2014 I taught physics skills at the Glasgow Physics Pre-University Summer School to about twenty recent high school graduates from impoverished backgrounds. My goal was to get them up to speed with physics skills needed for university study. I chose the following topics as most important for the five two hour classes assigned: researching scientific literature, algebra and trigonometry, dimensional analysis, error propagation, orders of magnitude. After covering the material, I took the initiative to prepare exercises as in a mock exam, very similar to what had already been solved, and then provide the step-by-step solutions after the class. The students appreciated this and felt it helped them grasp even better the different mathematical and physics skills needed to start a physics degree. 

%I used theory, exercices and solutions prepared by a studend specially hired by the university for this task, and was helped by two PhD students. The students were organised in three groups of about seven persons each, to encourage them to collaborate and those more advanced to teach those that needed help. We would then go through the tables and discuss their solutions and methods and help them. After a while I would write the correct solution on the board for everyone.

\subsection{Computing group project supervision}

In the springs of 2013, 2014, and 2015, at the University of Glasgow, I was one of the supervisors for the C++ computing group project for third year theoretical physics students. After having studied a Unix and C++ class, students were assigned to groups of about five persons and asked to solve the Laplace equation for physical systems. For simple systems, the Laplace equation was solved both analytically and numerically, thus validating the software. For any complex system, the software could then be used. I supervised weekly their progress and assessed the individual presentations and reports. I offered support for the group presentation and reports, but they were graded by senior faculty. The experience from the first year helped me improve the student's experience in the second year. I started by noting a variety in what questions were answered by different groups. Therefore, I wrote more precise instructions, which led to more structured analyses. Having also noticed a less than ideal quality in the presentations and reports, I asked this year that the reports are done in LaTex and that the structure of talks follows the guidelines of Jean-luc Dumont~\cite{JeanLucDumontVideo}~\cite{JeanLucDumontBook}, emphasing clear message over long descriptions. At the suggestion of the the senior supervisor, we also created all generic systems to be particle physics detectors, to relate more the abstract theory with real life examples. The students were then successful in finding paid summer internships in UK and abroad. 

\subsection{Teaching computing skills}

In the winter of 2012 and falls of 2013 and 2014, at the University of Glasgow, I taught two C++ lab and four ROOT labs for PhD studens in particle physics from across Scottish universities, thanks to the Scottish Universities Physics Alliance. C++ is the standard programming language used in particle physics, whereas ROOT is an open-source CERN-created C++-based library that is used in all particle physics analyses worldwide. ROOT performs efficiently data storage and retrieval, statistical analysis of data and making plots and graphs of the results. My goal is to have the students ready to start data analysis in particle physics for their PhD research. I chose a lab versus an official class as I believe especially in programming students learn by doing. I learned to design the curriculum and the exercises. I developed software exercise that work out of the box, to be compiled and run, for different examples. The main challenge was the fact that students came with different level of programming experiences. As each student was able to work at his or her own pace, with simple exercises to begin with, ensured that everyone learned something for the class. 

\subsection{PhD student mentoring}

As a postdoc at the University of Glasgow, I lead our university's effort on a Higgs boson search in the ATLAS experiment. As such, I supervised the day-to-day work of two PhD students, offering feedback on code, presentations and data analysis.  

\subsection{BSc student mentoring}

In the summer of 2013 at the University of Glasgow I supervised four undergraduate students offering them a research project. My goal was not so much their physics results at the moment, but to make sure they acquire the physics skills that would allow them to create great research down the road: computing skills, presentation skills, informal lecture introducing particle physics. One student contributed significantly to my research and the following year received a paid internship as a result. They all appreciated how I offered them clear examples of how research is done and presented, which they hadn't received before.

\ \\In the summer of 2012 also at the University of Glasgow I supervised a very bright and enthuastic high school student. The goal was to indentify a project that she would be able to tackle with no math or computing skills. Therefore, she drew photos of simulated collisions with many particles and spotted patterns that a computer was not be able to. 

\ \\In the summer of 2010 as a PhD student at the University of Glasgow I supervised an undergraduate student who continued one of my data analysis projects. 

\subsection{Homework and exam marking}

For six semesters at McGill Unversity I graded homeworks and exams for the large classes of first year mechanics, and electromagnetism, and optics. I learned that the best strategy for efficient and correct marking is to read first about five papers to have a feel of the quality of the solutions and the common mistakes. I would then create a marking scheme divided by items and respect it for every students. I would also first grade one problem for all students before moving at another problem. 

\subsection{Course tutoring}

For two semesters at McGill University I was a tutor for a physics course. I first had to solve the problems myself and then explain them on a board in front of the class, then grade their homeworks. I learned that ideally one would give time to the students to try the problems themselves first, but there was not enough time for that. If I were to organise the course and the problems set given to me by the lecturer, I would solve fewer problems, but in greater detail, to make sure all students were able to solve the basic problems, as a complex problem is formed from simpler problems. 

\subsection{Laboratory supervision}

For two semesters at McGill University I supervised the laboratories of first year mechanics and electromagnetism. Though there were instructions that students were supposed to follow to collect the data, I took the initiative to offer them a theoretical introduction at the beginning of each lab, as I realized that they were not well prepared theoretically. The students really appreciated that. 

\subsection{Private tutoring}

I have constantly been a private tutor of math and physics during my high school, BSc, MSc and PhD studies. I am very passionate about making difficult concepts clear to the students. Nothing is more rewarding than seeing the spark in their eyes when they grasp the understanding of a difficult concept. 

\subsection{Student representative at the Canadian Association of Physics}

For two years I represented graduated students in Canada at the Canadian Association of Physics Council that met twice a year. It allowed me to see the large picture of how a national association of physics is run and what initiatives it takes to allow for a smooth collaboration between national universities, as well as between students and academics. I was also local organiser for the Canada-America-Mexico physics graduate student conference in the summer of 2007 at McGill University, where I learned a lot about managing a team and organising a conference. 

\subsection{Science outreach}

Ever since the start of my MSc in particle physics in Canada, I have been passionate about communicating science in general and particle physics in particular to the general public or students and teachers of all ages. In March 2015 I gave a talk at TEDxUniversity of Glasgow about the Higgs boson discovery. In my home country, Romania, I created the first large scientific online community, persuaded major newspapers to create science sections, wrote hundreds of science news and outreach articles, trained dozens of high school studens in scientific journalism, was interviewed many times by the media, gave introductory talks on the subject of particle physics to about two dozen high schools and science events, and mentored science projects. In Canada I enthused primary school children about solar cells and renewable energy, and taught hands-on how to create recycled paper. In the US, UK, and Switzerland I was a volunteer at several particle physics exibitions at universities or labs such as Fermilab and CERN, using both English and French. All these activities and more are detailed in Appendix~\ref{appendix:ScienceOutreach}. 

\section{Evidence of teaching effectiveness} % Major section
Both students I supervised for the computing project lab or for undergraduate summer research expressed their appreciation that I offered a context for the projects and equipped them with the tools to do the research and then present their results in a way they had not been taught before. The two PhD students I supervise in their day-to-day research are making good progress and together we build a coherent small research group, producing innovative results for a Higgs boson search in the ATLAS experiment. Testimonials are present in Appendix~\ref{appendix:Testimonials}.

\section{Teaching development activities} % Major section
As a Teaching Assistant at McGill University, I helped every semester with teaching activities, such as labs, tutorials, and grading. In order to improve my teaching methods, I twice attended a workshop on modern teaching methods offered by the university (T-PULSE)~\cite{T-PULSE}, where I learned that traditional teaching is teacher-centred through monologue lectures, while the new effective teaching is student-centred through student engagement, which I have integrated in my teaching ever since.

\ \\When teaching labs or tutorials, I carefully read the course syllabus and material prepared by the mail lecturer, so that I would be able to teach it myself later on. 

\ \\I volunteered as a helper for a Software Carpentry workshop~\cite{SoftwareCarpentry}, where I learned better techniques of developing software for scientific research, which needs to meet the requirements of being reproducible and reusable. I now use those elements both in my own research and teaching.  

\ \\At the University of Glasgow I attended several workshops for staff development about improving my supervision, managing and collaboration skills. 

%%%%%%%%%%%%%%%%% 
\begin{appendices}
\section{Science outreach} \label{appendix:ScienceOutreach} % Major section

\subsection{Science outreach benefiting Romania}

During my MSc and PhD studies in Canada/US, I contributed at several levels to the science outreach in Romania. I wrote hundreds of science news and outreach articles, which were regularly published online by major newspapers in Romania, and printed by Romanian newspapers in Canada. At the time science sections were missing from Romanian newspapers. I took the initiative to contact them and offer my articles free of charge. Though reluctant at first, they were pleasantly surprised to observe there is a market for scientific articles, with an average of two thousand readers and one hundred comments. As scientists rarely take the time to communicate to the public, the readers expressed their gratitude for my responses to all their scientific questions and comments. As a result of my encouragement, a few other Romanian scientists started writing for the public. As a result, all major newspapers created permament scientific sections and some even hired dedicated staff or hosted science bloggers. Also, thanks to the very attractive topic of the Higgs boson search, I was interviewed many times. Due to all these efforts, more and better scientific information reached the Romanian general public.  

\ \\Another way I contributed to the Romanian society is by creating the largest scientific online community around my websites of particle physics (FizicaParticulelor.ro) and generic science (StiintaAzi.ro), with a combined audience of about 5,000 readers daily. The generic science website also hosted the largest science forum in the country, with many people interested in discussing science, or students seeking help with homework, or information about studying abroad. Besides my articles, the websites hosted those of about two dozens high school or university volunteers whom I recruited and trained in scientific journalism. Realizing that in our society a large majority of students plagiarise routinely, I emphasized the importance of honest writing and leading by example. For example, writing an article requires creating one's own story line from the original articles, which need to be referenced, while translations require a written approval, as offered to us by the Nobel Prize foundation, NASA, or the particle physics outreach magazines Symmetry from US and Elementaire from France, the Particle Adventure website or the LHSee smartphone app. The volunteers expressed their appreciation for my support and encouragement. I also organised three science essays competitions. The winners and the volunteers were rewarded with prizes in science books that I bought from the US with funds I raised from local businesses, as well as with recommendation letters when they applied for studying abroad, which most of them did successfully. 

\ \\I further contributed by giving introductory particle physics talks in about two dozens high schools across Romania and the Republic of Moldova, as well as at events such as Researcher's Night~\cite{NoapteaCercetatorilor}. 

\ \\At the end of my PhD studies, I transfered free of charge all the content, forum, contacts and expertise to  another science volunteering project that had started since, Scientia.ro, and sadly closed down the initial two websites. The legacy of this multifold project is a still ongoing network of science enthusiasts formed consisting of dozens of students, teachers, journalists. I continued to promote science in Romania by focusing  on particle physics, in articles, interviews, talks, and mentoring. A list of selected articles and interviews may be found at~\cite{BuzatuOutreach}.

\ \\During the last two years I was a mentor for the SmartNation.ro science and tech project lead by a major internet provider in Romania as part of their corporate responsibility. High school and university students were pitching science or tech projects to improve our every day life, ranging from smartphone apps to building robots. The best of them were funded to completion. I wrote science articles, offered feedback to students and was a member of the judging committee. Noting that students lacked training in effectively communicating the essence of the project, I designed a template that will be train the participants in the next year's edition. 

\subsection{Science outreach benefiting other countries}

As a volunteer at ``Let's talk science McGill''~\cite{LTSMcGill}, I made several visits to primary schools getting children to play with solar cells (discussing about renewable energies) and create recycled paper themselves. I saw first hand how engaging the students is the best way for them to learn. 

\ \\Also in Montreal I was a judge at a handful of high school level science fair competitions, where I learned skills on how to efficiently observe, ask questions and grade a project. 

\ \\I volunteered for three particle physics exibitions in the UK~\cite{PP4SS} and one at CERN~\cite{CERNOpenDays2013} in Switzerland dedicated to the general public. 

\ \\In March 2015 I gave a talk at the TEDxUniversityGlasgow event. 

%\newpage

\section{Testimonials} \label{appendix:Testimonials} % Major section

%\subsection{Undergraduate research internship}

\ \\Lynn Russel, summer student at the University of Glasgow, after she finished her fourth year at our university: ``I know I keep saying this but it really has been an amazing and invaluable experience. It has shown me just what I need to work on to get to where I want to be. Your support and advice has been incredibly important. Wishing all the best in the future! All the feedback has been extremely invaluable and for the first time I feel that I have been provided with the tools to correctly tackle a scientific report now. Thank you, thank you, thank you!''

\ \\Niveditha Ram, summer student at the University of Glasgow, after she finished her third year in India: ``Richard Feynman once said, "There is a difference between knowing something and knowing the name of something" I understood the full meaning of this at my summer internship with Dr. Adrian Buzatu at University of Glasgow. It was mainly a computational project. Although I had done a bit of programming in high school, I learnt here how a program should be written systematically, how to debug with ease, how to rethink at various stages if we can automate a particular task. Apart from this, his perfectionism, attention to detailing, patience to explain new ideas (be it about neural networks or the gold market) with different analogy are all something I enjoyed and learnt. Six weeks under his guidance left behind a strong impact and positive attitude to learn."

\ \\Kamen Petrov, summer student at the University of Glasgow, after he finished his second year at our university: ``In 2013 I participated in an internship lead by Dr. Adrian Buzatu, that aimed at creating a method for improving the mass resolution of the Higgs boson using artificial neural networks. Due to the nature of the project, the tasks that I had to do were 90\% programming with Python under Linux and 10\% physics, which is exactly what I loved about the internship. Over the course of the internship I gained a lot of knowledge and proficiency with an until then a relatively new to me programming language and operational system. A big factor in this was the fact that Adrian didn't only spend time with us to help us understand our tasks and get them done in a good way, but he spent time during lunch breaks and outside of work hours, to show us some other programming related projects that he was working on. Overall the entire internship for me was a learning experience but not only because of the tasks which I  was given but also because of the time which I spent with Adrian."

\ \\Paulin Todev summer student at the University of Glasgow, after he finished his second year at our university: `` started a summer project with Adrian due to my interest in both artificial neural networks and particle physics. Working under his supervision expanded my knowledge of program design, Python scripting and basic particle physics due to his quick and insightful support and attention to detail. This experience has been valuable in my daily studies as well as in finding funded positions for summer projects later on."

%\subsection{Group computing project supervision}

\ \\Augustinas Malinauskas, third year theoretical student, supervised for the computing group project in 2014: ``Adrian was my supervisor for Project Computational lab. During the project he was always guiding us towards solving the problem, yet, he never gave us the answer. Better - he tried to teach us how to approach the problem like it's done in real life science: starting from fundamental principles and building analogies to completely different (but mathematically analogous) processes like heat flow and diffusion. During the project I felt very motivated and spent hours discussing physics with my colleagues, with no doubt during the project I've learned more than in any other course. Adrian spent a lot of time helping us to prepare for presentations by giving tips, providing useful material and even lending personal book, I now have very different understanding of how to give a good presentation and that is something I'll use for the rest of my carrier. I think we had a wonderful supervisor for the project and learned a lot."

\ \\Ciaran Cruise, third year theoretical student, supervised for the computing group project in 2014: ``It is my pleasure to say that Dr Buzatu provided teaching and dedication to his position deserving
merit. Excellent selection of teaching resources, drawing from both credible academic sources and his
own personal experience in the working academic environment, facilitated good understanding
(amongst the participants) with the additional bonus of laying good practice for placements in
the academic work place. Also of note was his recommendation of resources dedicated to areas of study
previously under-developed in the degree program, e.g. public speaking and report writing, which places Dr Buzatu's teaching in the highest tier in terms of scope and utility. Always professional and concise in his appraisals, a most excellent supervisor." 

\ \\Matheus Sarmento, third year theoretical student, supervised for the computing group project in 2014: ``Adrian was my advisor during the discipline Theoretical Physics Project and he has taught in a interesting and inspiring way. He helped us to organize our groups activities, to prepare our presentation and to think about the physics process going on during the project. He was very enthusiastic about each new result that we have obtained and it inspired us to continue researching and giving ourselves more and more targets."

\ \\David Muir, third year theoretical student, supervised for the computing group project in 2014: ``I would say that the supervision of our physics project this year was well structured and organised. You provided valuable guidance with regards to what we had achieved and potential improvements that we could consider for the future. You were approachable and always willing to help. Your knowledge and guidance in all areas (physics theory, computer programming and presentation skills) was invaluable to me as a student."

\ \\Laurynas Mince, third year theoretical student, supervised for the computing group project in 2015: "I would like to express my very great appreciation to Dr. Adrian Buzatu, my supervisor, for his valuable and constructive suggestions during the third-year Theoretical Physics group project. His patient guidance, enthusiastic encouragement and useful critiques were very much appreciated and helped to improve the quality and presentation of the project."





\end{appendices}


\begin{thebibliography}{99} % Bibliography - this is intentionally simple in this template

\bibitem{T-PULSE}\url{http://www.mcgill.ca/tpulse/}
\bibitem{JeanLucDumontVideo}\url{http://www.youtube.com/watch?v=meBXuTIPJQk}
\bibitem{JeanLucDumontBook}\url{http://treesmapsandtheorems.com/}
\bibitem{SoftwareCarpentry}\url{http://software-carpentry.org}
\bibitem{NoapteaCercetatorilor}\url{http://noapteacercetatorilor.eu}
\bibitem{BuzatuOutreach}\url{http://www.adrianbuzatu.com/outreach.html}
\bibitem{LTSMcGill}\url{http://www.letstalkscience.ca/mcgill-university}
\bibitem{PP4SS}\url{http://www.scifun.ed.ac.uk/pp4ss}
\bibitem{CERNOpenDays2013}\url{http://opendays2013.web.cern.ch}
 
\end{thebibliography}

\end{document}

